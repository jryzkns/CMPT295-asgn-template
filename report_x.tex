\documentclass{article}
\usepackage[utf8]{inputenc}
\usepackage[margin=0.5in]{geometry} % lots more margin
\pagenumbering{gobble} % ignore page numbers


\title{CMPT 295 Assignment X}
\author{Ma Name Jeff}
\date{student number}

\usepackage{enumitem}
\usepackage{graphicx}
\usepackage{amsmath}
\usepackage{amsfonts}
\usepackage{hyperref} % for nice looking urls
\usepackage{booktabs} % for making tables
\usepackage{amssymb}
\usepackage{listings}
\usepackage{graphicx}
\usepackage{caption}


% lst settings for x86_86
% courtesy of https://tex.stackexchange.com/questions/51645/x86-64-assembler-language-dialect-for-the-listings-package
\lstdefinelanguage
   [x64]{Assembler}     % add a "x64" dialect of Assembler
   [x86masm]{Assembler} % based on the "x86masm" dialect
   % with these extra keywords:
   {morekeywords={CDQE,CQO,CMPSQ,CMPXCHG16B,JRCXZ,LODSQ,MOVSXD, %
                  POPFQ,PUSHFQ,SCASQ,STOSQ,IRETQ,RDTSCP,SWAPGS, %
                  rax,rdx,rcx,rbx,rsi,rdi,rsp,rbp, %
                  r8,r8d,r8w,r8b,r9,r9d,r9w,r9b, %
                  r10,r10d,r10w,r10b,r11,r11d,r11w,r11b, %
                  r12,r12d,r12w,r12b,r13,r13d,r13w,r13b, %
                  r14,r14d,r14w,r14b,r15,r15d,r15w,r15b,
                  imull}} % etc.

\lstset{language=[x64]Assembler}

% lst settings for C
% courtesy of https://tex.stackexchange.com/a/348653
\lstdefinestyle{CStyle}{
    basicstyle=\footnotesize,
    breakatwhitespace=false,         
    breaklines=true,                 
    captionpos=b,                    
    keepspaces=true,                 
    numbersep=5pt,                  
    showspaces=false,                
    showstringspaces=false,
    showtabs=false,                  
    tabsize=2,
    language=C
}

\begin{document}

\maketitle

\section{Awooo Nyah}
Yes.

\section{Numeric Codes}

Try stuff like this:

$$
\mathtt{0xffffffff} = \mathtt{1111\;1111\;1111\;1111\;1111\;1111\;1111\;1111}_2
$$

\section{Tables}
\begin{table}[h]
   \centering
   \begin{tabular}{|c||cc|cc|cc|cc|}
       \hline
       $N$ & $a$ & $b$& $c$ & $d$&$e$ & $f$&$g$ & $h$\\
       \hline
       512 &2.102&1.216&61.00&1.821&01.72&0221.90&01.71&0.89\\
       640 &2.812&1.411&7.117&1.951&1.137&1.111&1.36&11.11\\
       768 &6.101&1.813&161.110&2.513&21.50&1.316&2.148&1.35\\
       896 &18.90&2.1017&261.32&2.197&3.815&1.517&3.812&1.56\\
       1024 &181.40&12.64&90.212&14.48&5.719&11.77&5.518&1.77\\
       \hline
   \end{tabular}
   \caption{This is a caption\\This is the same caption on the second line}
\end{table}

\section{Sample Code}
The code for the following can be found in \texttt{kek.s}
\begin{lstlisting}

	.globl kek
kek:
	cmp $cross, $me # try crossing me
	je gtfo
gtfo:
	mov $f***, %here # get the f*** out of here
	ret

\end{lstlisting}
The code for the following can be found in the few inches below this line.
\begin{lstlisting}[style=CStyle]

#include<stdio.h>

int main(){
   int* x;
   *x = 5;
   printf("%d\n",*x);
   return 0;
}
   
\end{lstlisting}
\newpage

\section{Staged Execution}
\begin{table}[h]
   \centering
   \begin{tabular}{l|ccccccc}
   Instruction & $\mathtt{t_1}$ &$\mathtt{t_2}$ &$\mathtt{t_3}$ &$\mathtt{t_4}$ &$\mathtt{t_5}$ &$\mathtt{t_6}$ &$\mathtt{t_7}$ \\
   \hline
   \texttt{OP   \%reg, \%reg} & F & D & C & M & W\\
   \texttt{OP   \$1, \%reg} & & F & D & C & M & W\\
   \texttt{OP  \%reg, \%reg} & & & F & D & C & M & W
   \end{tabular}
\end{table}


\section{Arithmetic}
\begin{figure}[h]
   \centering
   \begin{tabular}{c*{27}{@{\enspace}c}}
      &0.&0&0&0&0&0&0&0&0&0&0&0&0&0&0&0&0&0&0&0&0&0&0&0&0& $\times 2^{a}$\\
      +&0.&0&0&0&0&0&0&0&0&0&0&0&0&0&0&0&0&0&0&0&0&0&0&0&0& $\times 2^{a}$\\
      \hline
      0&0.&0&0&0&0&0&0&0&0&0&0&0&0&0&0&0&0&0&0&0&0&0&0&0&0&$\times 2^{a}$
   \end{tabular}
   \caption{example of a floating point additon}

   \begin{tabular}{c*{8}{@{\enspace}c}}
      & $_{0}$ & $_{0}$ & $_{0}$ & $_{0}$ & $_{0}$ & $_{0}$ & $_{0}$ & $_{0}$\\ % carry bits
      & 0 & 0 & 0 & 0 & 0 & 0 & 0 & 0\\
    + & 0 & 0 & 0 & 0 & 0 & 0 & 0 & 0\\
    \hline
    \textbf{0} & 0 & 0 & 0 & 0 & 0 & 0 & 0 & 0\\
   \end{tabular}
   \caption{example of a binary addition, with carry digits}


\end{figure}

\section{Pseudocode}
\begin{verbatim}
   i = 0
   while yes:
           stuff happens
           using verbatim env allows for wild things like \begin{} not doing anything
   
\end{verbatim}

\end{document}
